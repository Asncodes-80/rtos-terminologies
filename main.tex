\documentclass[10pt, a4paper]{article}
\usepackage{geometry}
\usepackage{listings}
\usepackage{hyperref}
\usepackage{graphicx}
\usepackage{ragged2e}
\usepackage{color}
\usepackage{xepersian}
\usepackage{subfiles}
\newgeometry{left=1.4cm, right=1.4cm, bottom=2.0cm, top=2.0cm}
\settextfont[Scale=1]{XB Roya}
\usepackage{multirow}
\renewcommand{\baselinestretch}{1.5}
\definecolor{dkgreen}{rgb}{0,0.6,0}
\definecolor{gray}{rgb}{0.5,0.5,0.5}
\definecolor{mauve}{rgb}{0.58,0,0.82}
\definecolor{commentColor}{rgb}{0.6,0.6,0.60}
% Code style configuration
\lstset{frame=tb,
  language=python,
  aboveskip=1mm,
  lineskip=0.9mm,
  belowskip=1mm,
  showstringspaces=false,
  showspaces=false,
  columns=flexible,
  basicstyle={\small\ttfamily},
  numbers=none,
  keywordstyle=\color{mauve},
  commentstyle=\color{commentColor},
  stringstyle=\color{dkgreen},
  numberstyle=\small\color{black},
  numbers=left,
  stepnumber=1,
  breaklines=true,
  breakatwhitespace=true,
  tabsize=3
}

\begin{document}

تاریخ انتشار: \today

\textbf{اصطلاحاتی از مقاله \lr{Reliable Real-Time Operating System for IoT
Devices}}

نویسنده: علیرضا سلطانی نشان، دانشجوی ارشد مهندسی نرم‌افزار - دانشگاه آزاد اسلامی
واحد تهران-شمال \href{mailto:a.soltani@iau-tnb.ac.ir}{a.soltani@iau-tnb.ac.ir}



موضوعی که در ابتدا مطرح میکنه، در مورد فراگیر شدن گسترده دستگاه‌های مبتنی بر IoT
هستش که میگه از وسایل خانه گرفته تا مهم‌ترین وسایل پزشکی. به طوری که به صورت
گسترده در زندگی انسان‌ها در حال پیشرفت می‌باشند. 

نتیجه این برگه به طور کلی، ارزیابی محققان را بر سیستم عامل TizenRT نشان می‌دهد
که تسک‌هایی که حاوی خطا هستند را از فضای رم جدا نگهداری می‌کند در حالی که تضمین
اجرای بدون مشکل را برای تسک‌های Real-Time به صورت کامل می‌دهد که در مدت زمان
معینی که قرار است یک تسک کامل شود، انجام گیرد (در اینجا بهترین زمان برای انجام
تسک را ۵۰ میکروثانیه دیده اند). در ادامه به آن می‌پردازد، تسکی به خاطر خطا متوقف
شد چگونه می‌تواند به چرخه حیات مجدد خودش باز گردد؟ معرفی ویژگی Fast Recovery از
این سیستم عامل نشان دهنده آن است که بدون نیاز به Reboot کردن سیستم عامل می‌تواند
تسک مشک دار قبلی را در مرحله اجرای مجدد قرار داد (بهترین زمانی که محققان برای
ارزیابی در نظر گرتفن ۱۰ میلی ثانیه بوده است). به این دلیل است که سیستم عامل
TizenRT را انتخابی برای ماموریت‌های خاص (انجام تسک‌های حساس، مهم و بحرانی) معرفی
می‌کند.

این مقاله به طور کلی به دو مورد از ویژگی‌های اصلی که یک سیستم عامل Real Time
می‌پردازد:

ضعف اصلی برنامه نویس به دلیل پیچیدگی (در محیط و اشل‌های گسترده) نرم‌افزار می‌باشد.

\subsection{ویژگی‌ها}

\subsubsection{ویژگی \lr{Fault isolation}}

ویژگی \lr{Fault isolation} از اسمش معملومه، یعنی جداکننده خطا و فاجعه نرم‌افزاری
یک برنامه از دیگر برنامه‌ها. اگر یک برنامه دچار خطا شود، سیستم‌ عامل آن را به
صورت خودکار از برنامه‌های دیگر جدا می‌کند تا این حادثه بر اثر خرابی یک برنامه،
روی برنامه‌های دیگر تاثیر نگذارد. دلیل اصلی این ویژگی حضور \lr{Per-binary Memory
Protection} هستش که باید تو این بین بررسی بشه. در حقیقت مهمترین قابلیت این ویژگی
جلوگیری از عمل راه‌اندازی مجدد یا Rebooting است. (احتمالا توی مقاله منظور از
User binary اون نرم‌افزارهایی هستش که برنامه نویس در مد کاربر اونا رو اجرا
میکنه). توابع \lr{Fault handler} بالاترین اولویت را  در راه‌اندازی Thread‌ها
دارند.  برای داشتن همچین قابلیتی نیازمند آن هستیم که قابلیت‌های Real-Time را در
سیستم به ذاتی داشته باشیم (یا حتی به وجود آورده باشیم).

\subsubsection{ویژگی \lr{Fast Recovery}}

در مقابل ویژگی به نام \lr{Fast recovery} وجود دارد که به برنامه کمک می‌کند در
مدت زمانی بسیار معقول و سریع، برنامه‌ای که با شکست مواجه شده‌ است را ریلود و
مجددا اجرا کند که بتواند به ادامه فرایند محاسباتی خودش بپردازد. مکانیزمی که برای
\lr{Fast recovery} پیاده‌سازی شده است که از مرتبه و اولویت پایین‌تری نسبت به
Treadهای Real-Time برخوردار است. این عملیات به گونه‌ای انجام می‌شود که عملکرد
برنامه‌های حساس دیگر  را تحت تاثیر قرار ندهند.

\subsection{موضوع \lr{Per-binary Memory Protection}}

در این قسمت صد درصد مطمئن شدم که منظور از Binary همون \lr{Executable Program}ها
می‌باشد. قابلیتی در سیستم عامل‌ها و پردازنده‌های مدرن و امروزی است که به
برنامه‌ها اجازه میدهند که به صورت انفرادی دسترسی به مموری خودشان داشته باشند و
آن را به صورت کاملا مستقل کنترل و محافظت کنند. این بدان معناست که هر برنامه در
حال اجرا می‌تواند مجموعه‌ای از دسترسی‌ها و محدودیت‌های منحصر به فرد خودش را
داشته باشد. مثلا تا چه حدی می‌تواند به حافظه خودش دسترسی داشته باشد. اگر
برنامه‌ای تلاش کند که به مجوزی که برای اون نیست دسترسی داشته باشد از آن جلوگیری
می‌شود. این قابلیت باعث می‌شود تا برنامه روی مموری‌های یکدیگر دخالت نداشته
باشند. این نوع محافظت از حافظه، از مهمترین قابلیت‌های امنیتی در کامپیوتر است،
زیرا از نفوذ بدافزار‌ها و آسیب پذیری‌هایی که از طریق دسترسی به حافظه عمل
می‌کنند، جلوگیری می‌کند.


\subsection{اجرای همزمان تسک‌های \lr{RT} و \lr{NRT}}

سیستم عامل TizenRT می‌تواند تمام تسک‌های RT و NRT را با توجه به دو ویژگی
ایزوله‌سازی خطا و بازیابی سریع، به صورت همزمان اجرا کند.



\end{document}